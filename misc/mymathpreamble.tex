%%% PACKAGE IMPORTS

%% packages required for this preamble
\usepackage{xparse}

%% math packages
\usepackage{amsmath}
\usepackage{amsfonts}
\usepackage{amssymb}
\usepackage{amsthm}
\usepackage{esint}  % more integrals
\usepackage{cancel}  % cancellations
% \usepackage{accents}
\usepackage{bm}  % for bold vectors
% \usepackage{mathrsfs}
\usepackage[scr]{rsfso}

%% graphics-related packages
\usepackage{graphicx}
\graphicspath{ {./images/} }
% \usepackage{tikz}
\usepackage{float}
\usepackage{svg}

%% text-related packages
\usepackage{ulem}  % for striked-out text
\normalem  % ulem package changes emph to underline, revert change here
\usepackage[nodayofweek,level]{datetime}
\usepackage{parskip}
\usepackage{siunitx}
\sisetup{quotient-mode=fraction} % Output a/b as \frac{a}{b}
\sisetup{output-complex-root=\ensuremath{\mathrm{j}}}  % change sqrt(-1) to display as j instead of i

% need to usepackage hyperref at the very end


%%% THEOREM-LIKE ENVIRONMENT DEFINITIONS
% this makes it possible to link to a theorem, proposition or definition 
% if the begin statement has the optional name argument in [].

\theoremstyle{plain}
\newtheorem{inttheorem}{Theorem}[section]
\newtheorem{intproposition}{Proposition}[section]
\newtheorem{intdefinition}{Definition}[section]

\ExplSyntaxOn
    \DeclareDocumentEnvironment{theorem} { o } {
        \IfNoValueTF{#1} {
            \begin{inttheorem}
        } {
            \begin{inttheorem}[#1]
            \label{thm:#1}
        }
    } {
        \end{inttheorem}
    }

    \DeclareDocumentEnvironment{proposition} { o } {
        \IfNoValueTF{#1} {
            \begin{intproposition}
        } {
            \begin{intproposition}[#1]
            \label{prop:#1}
        }
    } {
        \end{intproposition}
    }

    \DeclareDocumentEnvironment{definition} { o } {
        \IfNoValueTF{#1} {
            \begin{intdefinition}
        } {
            \begin{intdefinition}[#1]
            \label{def:#1}
        }
    } {
        \end{intdefinition}
    }
\ExplSyntaxOff

%% other theorem-like environment definitions which don't require labelling
\theoremstyle{plain}
\newtheorem{lemma}{Theorem}[section]
\newtheorem{corollary}{Theorem}[section]

\theoremstyle{remark}
\newtheorem*{remark}{Remark}


%%% MATHEMATICAL SYMBOLS

%% Operators
% calculus
\let\div\relax  % just get rid of the old division symbol
\DeclareMathOperator{\div}{div}
\DeclareMathOperator{\curl}{curl}
% analysis
\DeclareMathOperator{\usum}{U}
\DeclareMathOperator{\lsum}{L}
% linear algebra
\DeclareMathOperator{\spn}{span}
\DeclareMathOperator{\tr}{tr}
\DeclareMathOperator{\proj}{Proj}
\DeclareMathOperator{\col}{Col}
\DeclareMathOperator{\row}{Row}
\DeclareMathOperator{\rowech}{ref}
\DeclareMathOperator{\rank}{rank}
\DeclareMathOperator{\nullspace}{NS}
% hyperbolic trig functions
\DeclareMathOperator{\arsinh}{arsinh}
\DeclareMathOperator{\arcosh}{arcosh}
\DeclareMathOperator{\artanh}{artanh}

%% brackets
\newcommand{\abs}[1]{\left\lvert #1 \right\rvert}
\newcommand{\inner}[2]{\left\langle #1 ,\, #2 \right\rangle}
\newcommand{\floor}[1]{\left\lfloor #1 \right\rfloor}
\newcommand{\angleb}[1]{\left\langle #1 \right\rangle}
\newcommand{\norm}[1]{\left\| #1 \right\|}

%% vector calculus
% tilde underset as vector notation - i don't use anymore
% \DeclareMathAccent{\wtilde}{\mathord}{largesymbols}{"65}
% \newcommand{\tildevec}[1]{\underaccent{\wtilde}{{#1}}}

\let\oldvec=\vec  % old arrow accent symbol
\renewcommand{\vec}[1]{\bm{\mathbf{#1}}}  % https://tex.stackexchange.com/a/3544
\newcommand{\mat}[1]{\vec{#1}}  % for now

% https://tex.stackexchange.com/a/188777
\newcommand{\uveci}{{\bm{\hat{\textnormal{\bfseries\i}}}}}
\newcommand{\uvecj}{{\bm{\hat{\textnormal{\bfseries\j}}}}}

\NewDocumentCommand{\uvec}{mo} {
    \ifcat\relax\noexpand#1%
        \bm{\hat{#1}}
    \else
        \ifcsname uvec#1\endcsname
            \csname uvec#1\endcsname
        \else
            \bm{\hat{\mathbf{#1}}}
        \fi
    \fi
    \IfNoValueF{#2} {
        _{\boldsymbol{\mathbf{#2}}}% please work?
    }
}

% \DeclareRobustCommand{\uvec}[1]{{%
%   \ifcat\relax\noexpand#1%
%     % it should be a Greek letter
%     \bm{\hat{#1}}%
%   \else
%     \ifcsname uvec#1\endcsname
%       \csname uvec#1\endcsname
%     \else
%       \bm{\hat{\mathbf{#1}}}%
%      \fi
%    \fi
% }}

\newcommand{\grad}{\vec{\nabla}}

%% basic calculus
\newcommand{\deriv}{\,\mathrm{d}}
\newcommand{\derivfrac}[2]{\frac{\deriv #1}{\deriv #2}}
\newcommand{\pfrac}[2]{\frac{\partial #1}{\partial #2}}

\newcommand{\uint}[2]{\displaystyle{\int^{\overline{#2}}_{#1}}}
\newcommand{\lint}[2]{\displaystyle{\int^{#2}_{\underline{#1}}}}	
\newcommand{\nint}[2]{\displaystyle{\int^{#2}_{#1}}}

\newcommand{\laplace}{\mathscr{L}}

% \newcommand{\infsum}[1]{\sum \limits^{\infty}_{#1}}  % deprecated

%% linear algebra
\newcommand{\tpose}{\mathrm{T}}

%% other symbols
\newcommand{\degree}{^{\circ}}


%%% MISCELLANEOUS

%% custom units
\DeclareSIUnit \voltampere {VA}  % apparent power
\DeclareSIUnit \vareactive {VAR}  % reactive power
