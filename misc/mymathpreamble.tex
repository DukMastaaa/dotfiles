\usepackage{amsmath}
\usepackage{amsfonts}
\usepackage{amssymb}
\usepackage{amsthm}
\usepackage{float}
\usepackage{cancel}
\usepackage{graphicx}
\graphicspath{ {./images/} }
\usepackage{siunitx}
\usepackage{accents}

% Operators
\newcommand{\deriv}{~\mathrm{d}}
\DeclareMathOperator{\usum}{U}
\DeclareMathOperator{\lsum}{L}
\DeclareMathOperator{\spn}{span}
\DeclareMathOperator{\tr}{tr}
\DeclareMathOperator{\proj}{Proj}
\DeclareMathOperator{\col}{Col}
\DeclareMathOperator{\row}{Row}
\DeclareMathOperator{\rowech}{ref}
\DeclareMathOperator{\rank}{rank}

% Environments
\theoremstyle{plain}
\newtheorem{theorem}{Theorem}[section]
\newtheorem*{theorem*}{Theorem}

\theoremstyle{plain}
\newtheorem{lemma}{Theorem}[section]

\theoremstyle{plain}
\newtheorem{proposition}{Proposition}[section]

\theoremstyle{definition}
\newtheorem{definition}{Definition}[section]

\theoremstyle{remark}
\newtheorem*{remark}{Remark}

\theoremstyle{plain}
\newtheorem{corollary}{Theorem}[section]

% Left and right brackets
\newcommand{\abs}[1]{\left\lvert #1 \right\rvert}
\newcommand{\inner}[2]{\left\langle #1 ,\, #2 \right\rangle}
\newcommand{\floor}[1]{\left\lfloor #1 \right\rfloor}
\newcommand{\angleb}[1]{\left\langle #1 \right\rangle}
\newcommand{\norm}[1]{\left\| #1 \right\|}

% Other symbols
\newcommand{\utilde}[1]{\underset{\sim}{#1}}
\newcommand{\uvec}[1]{\utilde{#1}}
\newcommand{\grad}{\mathbf{\nabla}}
\newcommand{\degree}{^{\circ}}
	
% Calculus stuff
\newcommand{\derivfrac}[2]{\frac{\deriv #1}{\deriv #2}}
\newcommand{\pfrac}[2]{\frac{\partial #1}{\partial #2}}

\newcommand{\uint}[2]{\displaystyle{\int^{\overline{#2}}_{#1}}}
\newcommand{\lint}[2]{\displaystyle{\int^{#2}_{\underline{#1}}}}	
\newcommand{\nint}[2]{\displaystyle{\int^{#2}_{#1}}}

\newcommand{\infsum}[1]{\sum \limits^{\infty}_{#1}}

% Stuff for other packages
\sisetup{quotient-mode=fraction} % Output a/b as \frac{a}{b}